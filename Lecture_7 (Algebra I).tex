\documentclass{report}
\usepackage{amssymb}
\usepackage{mathtools}
\usepackage{amsmath}
\usepackage[utf8]{inputenc}
\usepackage[x11names]{xcolor}
\usepackage[textwidth = 16cm]{geometry}
\usepackage{tikz}
\usetikzlibrary{calc}
\usepackage{paracol}
\usepackage{microtype}
\usepackage{hyperref}
\newcounter{testexample}
\usepackage{xparse}
\usepackage{fontspec}
\setlength{\columnseprule}{0.2pt}
\usepackage{enumitem}
\usepackage{lmodern}
\usepackage{amssymb}% http://ctan.org/pkg/amssymb
\usepackage{pifont}% http://ctan.org/pkg/pifont
\newcommand{\xmark}{\ding{55}}%
\usepackage[makeroom]{cancel}
\usepackage[most]{tcolorbox}
\usepackage{titlesec}
\usepackage{newunicodechar}
\usepackage{tikz}\usetikzlibrary{shapes.misc}
\newcommand\xdownarrow[1][2ex]{%
   \mathrel{\rotatebox{90}{$\xleftarrow{\rule{#1}{0pt}}$}}
}
\newcommand*\circled[1]{\tikz[baseline=(char.base)]{
            \node[shape=circle,draw,inner sep=2pt] (char) {#1};}}
\usepackage{bm}
\newcommand\bigcdot{\bm{\cdot}}
\usepackage{booktabs}
\setlength\defaultaddspace{0.5ex}
\usepackage[math]{cellspace}
\setlength\cellspacetoplimit{3pt}
\setlength\cellspacebottomlimit{3pt}


\usepackage{accents}

\newcommand\titlebar{%
\tikz[baseline,trim left=3.1cm,trim right=3cm] {
    \fill [cyan!25] (2.5cm,-1ex) rectangle (\textwidth+3.1cm,2.5ex);
    \node [
        fill=cyan!60!black,
        anchor= base east,
        rounded rectangle,
        minimum height=3.5ex] at (3cm,0) {
        \textbf{\arabic{chapter}.\thesection.}
    };
}%
}
\titleformat{\section}{\large}{\titlebar}{0.1cm}{}
\renewcommand*{\thesection}{\arabic{section}}
\usepackage{graphicx}
\graphicspath{ {./images/} }

\def\particulartemplate#1{
  \begin{tikzpicture}[overlay, remember picture] 
    \draw let \p1 = (current page.west), \p2 = (current page.east) in
      node[minimum width=\x2-\x1, minimum height=2cm, draw, rectangle, fill=blue!20, anchor=north west, align=left, text width=\x2-\x1] at ($(current page.north west)$) {\Large\bfseries \quad #1};
  \end{tikzpicture}
}
\newenvironment{Definition}% environment name
{% begin code
  \par\vspace{\baselineskip}\noindent
  \textbf{\underline{\large{Definition:}}}\begin{itshape}%
  \par\vspace{\baselineskip}\noindent\ignorespaces
}%
{% end code
  \end{itshape}\ignorespacesafterend
}
\newenvironment{Theorem}% environment name
{% begin code
  \par\vspace{\baselineskip}\noindent
  \textbf{\underline{\large{Theorem:}}}\begin{itshape}%
  \par\vspace{\baselineskip}\noindent\ignorespaces
}%
{% end code
  \end{itshape}\ignorespacesafterend
}
\newcommand{\newbrace}[1][]{
\begin{tikzpicture}[baseline=-0.5ex]
\draw[#1] (0,0) -- (0.9,0.9);
\draw[#1] (0,0) -- (0.9,-0.9);
\end{tikzpicture}}
\newenvironment{casesnew}[1][->]%
{\;\newbrace[#1]\;\begin{array}{@{}l@{}}}%
{\end{array}}
\newunicodechar{ā}{\=a}
\newunicodechar{ṇ}{\d{n}}
\definecolor{pearl}{rgb}{0.94, 0.92, 0.84}

\newcommand{\mybox}[4]{
    \begin{figure}[h]
        \centering
    \begin{tikzpicture}
        \node[anchor=text,text width=\columnwidth-1.2cm, draw, rounded corners, line width=1pt, fill=#3, inner sep=5mm] (big) {\\#4};
        \node[draw, rounded corners, line width=.5pt, fill=#2, anchor=west, xshift=5mm] (small) at (big.north west) {#1};
    \end{tikzpicture}
    \end{figure}
}



  
\def\exampletext{Example} % If English

\NewDocumentEnvironment{testexample}{ O{} }
{
\colorlet{colexam}{red!55!black} % Global example color
\newtcolorbox[use counter=testexample]{testexamplebox}{%
    % Example Frame Start
    empty,% Empty previously set parameters
    title={\exampletext: #1},% use \thetcbcounter to access the testexample counter text
    % Attaching a box requires an overlay
    attach boxed title to top left,
       % Ensures proper line breaking in longer titles
       minipage boxed title,
    % (boxed title style requires an overlay)
    boxed title style={empty,size=minimal,toprule=0pt,top=4pt,left=3mm,overlay={}},
    coltitle=colexam,fonttitle=\bfseries,
    before=\par\medskip\noindent,parbox=false,boxsep=0pt,left=3mm,right=0mm,top=2pt,breakable,pad at break=0mm,
       before upper=\csname @totalleftmargin\endcsname0pt, % Use instead of parbox=true. This ensures parskip is inherited by box.
    % Handles box when it exists on one page only
    overlay unbroken={\draw[colexam,line width=.5pt] ([xshift=-0pt]title.north west) -- ([xshift=-0pt]frame.south west); },
    % Handles multipage box: first page
    overlay first={\draw[colexam,line width=.5pt] ([xshift=-0pt]title.north west) -- ([xshift=-0pt]frame.south west); },
    % Handles multipage box: middle page
    overlay middle={\draw[colexam,line width=.5pt] ([xshift=-0pt]frame.north west) -- ([xshift=-0pt]frame.south west); },
    % Handles multipage box: last page
    overlay last={\draw[colexam,line width=.5pt] ([xshift=-0pt]frame.north west) -- ([xshift=-0pt]frame.south west); },%
    }
\begin{testexamplebox}}
{\end{testexamplebox}\endlist}


\begin{document}
\vspace{2cm}


\vspace{\medskipamount}
\hrule height 0.4pt
\vspace{3pt}
\hrule height 1.6pt
    \particulartemplate{Lecture 7 Notes (Dr. Khaled Adarbeh) / Modern Algebra(I) }
 
 \vspace{2cm}

\begin{tcolorbox}[breakable, enhanced, sharp corners, colback=white!30, colframe=green!80!blue, title="Another Subgroup Test"]
\begin{Theorem}
Let $H$ , $K$ be a subgroup of a group $G$
\begin{enumerate}
\item $H \cap K$ is a subgroup.
\item $H \cup K$ is a subgroup \; $\Longleftrightarrow \; H \; \subseteq \; K$ or $K \; \subseteq \; H$
\item $\{hk / h \; \in \; H, k \; \in \; K\}$
\end{enumerate}
\end{Theorem}
\end{tcolorbox}
\emph{\underline{\textcolor{red}{Proof:}}}\\
\vspace{0.2cm}\\
\circled{1}\\
\vspace{0.2cm}\\
\underline{$1 \; \in \; H \cap K:$}\\ \vspace{0.2cm}

$\begin{rcases}
   1 \; \in \; K \; (\text{Subgroup})\\
   1 \; \in \; H \; (\text{Subgroup})\\
\end{rcases} \; \; \Longrightarrow \; 1 \; \in \; H \cap K$\\
$\begin{rcases}
x , y \; \in \; H \cap K\; \; \Longrightarrow \;   \begin{rcases}
   x \; \in \; H \; \\
   y \; \in \; H \; \\
\end{rcases} \; \; \Longrightarrow \; \; xy \; \in \; H \\ 
       \boxed{\text{and}} \hspace{2.3cm}  \begin{rcases}
   x \; \in \; K \; \\
   y \; \in \; K \; \\
\end{rcases} \; \; \Longrightarrow \; \; xy \; \in \; K
\end{rcases} \; \; \Longrightarrow \; \; xy \; \in \; H \cap K$\\
$x \; \in \; H \cap  K \; \; \Longrightarrow \; \; \begin{rcases}
x \; \in \; H \; \Longrightarrow \; \; x^{-1} \; \in \; H\\
   x \; \in \; K \; \Longrightarrow \; \; x^{-1} \; \in \; K \\
\end{rcases} \; \; \Longrightarrow \; x^{-1} \; \in \; H \cap K$\\
\vspace{0.3cm}\\
\circled{2}\\
\vspace{0.2cm}\\
"$\Longleftarrow$" \hspace{0.5cm} Trivial\\
$H \; \subseteq \; K \; \; \Longrightarrow \; H \cup K \; = \; K$, where $K$ is subgroup \\
$\therefore \; \; H \cup K$ is subgroup\\
\vspace{0.2cm}\\
"$\Longrightarrow$" \; \; \; Assume that:
\begin{itemize}
\item $H  \cup K$ is subgroup
\item $H \not\subseteq K$ \; \; (i.e. $\exists \; h_0 \; \in \; H$ \ $k$) 
\end{itemize}
we want to show $K \; \subseteq \; H$ :\\
$\begin{rcases}
\text{Let} \; k_0 \; \in \; K \; (\text{arbitrary}) \; \; \Longrightarrow \; k_0 \; \in \; H \cup K\\
\hspace*{0.5cm} h_0 \; \in \; K \; \; \; \; \; \Longrightarrow \; \; h_0 \; \in \; H \cup K
\end{rcases} \; \; \Longrightarrow \; k_0 h_0 \; \in \; H \cup K$ "because $H \cup K$ subgroup"\\
\vspace{0.1cm}\\
$k_0 h_0 \; \in \; H$ \hspace{1cm} or \hspace{1cm} $k_0 h_0 \; \in \; K$\\
$H$ is subgroup \hspace{2cm} $K$ is subgroup\\
$\Longrightarrow \; h_0^{-1} \; \in \; H$ \hspace{2.2cm} $\Longrightarrow \; k_0^{-1} \; \in \; H$ \\
$\Longrightarrow \; kh_0h_0^{-1} \; \in \; H$ \hspace{1.6cm} $\Longrightarrow \; k_0^{-1}kh_0 \; \in \; k$ \\
$\Longrightarrow k_0 \; \in \; H$ \hspace{2.5cm} $\Longrightarrow \; h_0 \; \in \; k$ \\
\newline
\hspace*{0.5cm} we have a contradiction. \\
"$ H \; \not\subseteq \; K \; \; i.e. \; \; \exists \; h_0 \; \in \; H \; \underbrace{/}_{\text{except}} \; K$\\
Thus $h \cup K$ is subgroup $\Longleftrightarrow \; \; H \; \subseteq \; K \; \; \text{or} \; \; K \; \subseteq \; H$\\
\newline
\emph{\underline{\textcolor{red}{Note:}}}\\
$H \cup K$ \; is  subgroup in general?\\
No.\\

\begin{tcolorbox}[breakable, enhanced, sharp corners, colback=white!30, colframe=green!80!blue, title="Center of a group"]
\begin{Definition}
$z(G) \; \colon= \; \{a \; \in \; G / \; ax \; = \; xa \; \; \; \forall \; x \; \in \; G\}$
\end{Definition}
\end{tcolorbox}
\emph{\underline{\textcolor{red}{Note:}}}\\
this is for the whole group.\\
\vspace{0.3cm}\\


\begin{tcolorbox}[breakable, enhanced, sharp corners, colback=white!30, colframe=green!80!blue]

\begin{Theorem}
$z(G)$ is an abelian subgroup of $G$
\end{Theorem}
\end{tcolorbox}

\emph{\underline{\textcolor{red}{Proof:}}}\\
\vspace{0.1cm}\\
\hspace*{0.3cm} \underline{$1 \; \in \; z(G)$ \; ???}\\
\vspace{0.06cm}\\
$1 . x \; = \; x . 1 \; = \; x$ \; \; it is true  \; \; $\therefore \; 1 \; \in \; z(G)$\\
\newline
\hspace*{0.3cm} \underline{$ a , b \; \in \; z(G) \; \xRightarrow{\text{??}} \; \; ab \; \in \; z(G)$}\\
\vspace{0.06cm}\\
$ax \; = \; xa$\\
$bx \; = \; xb$\\
$(ab) (x) \; = \; a (bx)$ \; \; \; "$b$ in center" \hfill "Asspciative"\\
\hspace*{0.5cm} $= a (x b)$ \hfill $b \; \in \; z(G)$\\
\hspace*{0.5cm} $= (a x ) b $ \hfill "Associative"\\
\hspace*{0.5cm} $= x (a b)$ \hfill $a \; \in \; z(G)$
\newline
\hspace*{0.3cm} \underline{$a \; \in \; z(G) \; \xRightarrow{\text{??}} \; \; a^{-1} \; \in \; z(G)$}\\
\vspace{0.06cm}\\
$ax \; = \; xa \; \; \; \; \; \forall \; x \; \in \; G$\\
$(ax \; = \; xa) \times (a^{-1})$\hfill "we multiply from both right and left"\\
$\Longrightarrow \; a^{-1} a x a^{-1} \; = \; a^{-1} x a a^{-1}$ \\
$\Longrightarrow \; e x a^{-1} \; = \; a^{-1} x e$\\
$\Longrightarrow \; x a^{-1} \; = \; a^{-1} x$ \; \; \; \; $\forall \; x \; \in \; G$\\
\newline
\hspace*{0.3cm} \underline{Abelian ??}\\
\vspace{0.06cm}\\
$ab \; = \; ba \; \; \; \; \forall \; a , b \; \in \; z(G) \; \Longrightarrow \; z(G)$ \; Abelian.\\
\vspace{0.2cm}\\
\emph{\underline{\textcolor{red}{Note:}}}
\begin{itemize}
\item $z(G)$ is maximum with this property ( Abelian) meaning that, $z(G)$ is the largest subgroup we can obtain from $G$
\item if $G$ is abelian group $\Longrightarrow$ \; \; $z(G) = G$
\end{itemize}

\begin{tcolorbox}[breakable, enhanced, sharp corners, colback=white!30, colframe=green!80!blue, title="Centralizer of an element"]
\begin{Definition}
$G$ group , $a \; \in \; G$\\
$C(a) \; = \; \{x \; \in \; G \; / \; ax \; = \; xa \}$
\end{Definition}
\end{tcolorbox}

\emph{\underline{\textcolor{red}{Note:}}}\\
\begin{itemize}
\item $C(a)$ this is for each element in the group.
\item $z(G) \; \subseteq \; C(a) \; \; \; \forall \; a \; \in \; G$
\item Centralizer is greater than center of a group.
\end{itemize}
\newpage
\mybox{Note}{red!40}{red!10}{\begin{displaymath} z(G) \; = \; \bigcap_{a \in G} C(a)\end{displaymath}}

\emph{\underline{\textcolor{red}{Proof:}}}\\
\vspace{0.03cm}\\
"\textcolor{red}{$\Longrightarrow$}" \; \; $\subseteq$\\
$x \; \in \; z(G) \; \Longrightarrow \; xa \; = \; ax$\\
\hspace*{1.6cm} $\Longrightarrow \; x \; \in \; C(a)$\\
\hspace*{1.6cm} $\Longrightarrow \; x \; \in \; \bigcap_{a \in G} \; C(a)$\\
\vspace{0.1cm}\\
"\textcolor{red}{$\Longleftarrow$}" \; \; $\supseteq$\\
$x \; \in \; \bigcap_{a \in G} \; C(a) \; \Longrightarrow \; x \; \in \; C(a) \; , \; \; \forall \; a \; \in \; G$\\
\hspace*{2.6cm} $\Longrightarrow \; xa \; = \; ax \;, \; \forall \; a \; \in \; G$\\
\hspace*{2.6cm} $\Longrightarrow \; x \; \in \; z(G)$\\
\vspace{0.1cm}\\
\emph{\underline{\textcolor{red}{Note:}}}\\
\begin{itemize}
\item $C(a)$ is a subgroup
\end{itemize}
\begin{testexample}
\textcolor{red}{Is $C(a)$ Abelian?}\\
\emph{\underline{\textcolor{red}{Solution:}}}\\
let $G$ a non abelian group \\
$e \; \in \; G \; , \; ea \; = \; ae \;= \; a$\\
\hspace*{1.2cm} $C(e) \; = \; a \; = \; G$ \hfill "But, $G$ is non abelian from the assumption above"\\
$\therefore \; C(e)$ \; non abelian 
\end{testexample}
\vspace{0.1cm}

\emph{\underline{\textcolor{red}{Note:}}}\\
\begin{itemize}
\item $G$ abelian \; $\Longleftrightarrow \; C(a) \; = \; G \; , \; \forall \; a \; \in \; G$\\
\hspace*{1.7cm} $\Longleftrightarrow \; z(G) \; = \; G$
\end{itemize}
\vspace{0.2cm}

\begin{testexample}
\boxed{\emph{\underline{\textcolor{red}{Ex: 37}}}}\\
\includegraphics[width=10cm, height=5cm]{ex37}\\
\vspace{0.2cm}\\
$C(1) \; = \; G$\\
$C(2) \; = \; \{1 , 2 , 5 , 6\}$\\
$C(3) \; = \; \{1 , 3 , 5 , 7\}$\\
$C(4) \; = \; \{1 , 4 , 5  , 8\}$\\
$C(5) \; = \; \{1 , 5, 2 , 3 , 4 , 6 , 7 , 8\}$\\
$C(6) \; = \; \{1 , 6 , 5 , 2\}$\\
$C(7) \; = \; \{1 , 7 , 3 , 5\}$\\
$C(8) \; = \; \{1 , 8 , 4 , 5\}$\\
\newline
$z(G) \; = \; \bigcap_{\forall a} \; C(a) \; = \; \{1 , 5\}$ \hfill "The intersection of all above sets"\\
\newline
\emph{\underline{\textcolor{red}{Note:}}} \; $5^2 \; = \; 1$
\end{testexample}
\begin{testexample}
\boxed{\emph{\underline{\textcolor{red}{Ex: 79}}}}\\
\newline
\textcolor{red}{$G \; = \; GL (2 , \mathbb{R})$}\\
\textcolor{red}{Find:}\\
\vspace{0.1cm}\\
1) \; $C\left(\begin{bmatrix}
1 & 1\\
1 & 0
\end{bmatrix}\right)$ \hspace{1cm} 2) \; $C \left(\begin{bmatrix}
0 & 1\\
1 & 0
\end{bmatrix}\right)$ \hspace{1cm} 3) $2(G)$\\
\vspace{0.2cm}\\
\emph{\underline{\textcolor{red}{Solution:}}}
\begin{enumerate}
\item $\begin{bmatrix}
a & b\\
c & d
\end{bmatrix} \; \begin{bmatrix}
1 & 1\\
1 & 0
\end{bmatrix} \; = \; \begin{bmatrix}
1 & 1\\
1 & 0
\end{bmatrix} \; \begin{bmatrix}
a & b\\
c & d
\end{bmatrix}$\\
\vspace{0.3cm}\\
$\begin{bmatrix}
a+b & a\\
c+d & c
\end{bmatrix} \; = \; \begin{bmatrix}
a+c & b+d\\
a & b
\end{bmatrix}$\\
\vspace{0.3cm}\\
$\Longrightarrow \; a + b \; = \; a +c \; \Longrightarrow \; b \; = \; c$\\
$\Longrightarrow \; a \; = \; b + d \; \Longrightarrow \; d \; = \; a - b$\\
$\Longrightarrow \; c + d \; = \; a$\\
$\Longrightarrow \; b + d \; = \; c + d \; \Longrightarrow \; b \; = \; c$\\
\vspace{0.1cm}\\
$\therefore \; \begin{bmatrix}
a & b\\
b & a-b
\end{bmatrix}$ \; \; \; \; \; \; \; \; s.t. \; $a(a - b) - b^2 \; \not= \; 0$\\
\vspace{0.1cm}\\
because we want to make sure it belongs to $GL ( 2 , \mathbb{R})$ \; so \; $det\left(\begin{bmatrix}
a & b\\
b & a-b
\end{bmatrix}\right) \; \not= \; 0$\\
$\Longrightarrow \; a^2 - ab - b^2 \; \not= \; 0$\\
So, \; $C\left(\begin{bmatrix}
1 & 1\\
1 & 0
\end{bmatrix}\right) \; = \; \left\{\begin{bmatrix}
a & b\\
b & a - b
\end{bmatrix} \bigg/ \; a^2 - ab - b^2 \; \not= \; 0\right\}$\\
\vspace{0.2cm}\\
\item \textcolor{red}{Answer is:}\\
\vspace{0.2cm}\\
$\left\{\begin{bmatrix}
a & b\\
b & a
\end{bmatrix} \; \bigg/ \; a^2 - b^2 \; \not= \; 0 \right\}$\\
\vspace{0.2cm}\\
\item Since $z(G) \;  = \; C_1 \; \cap \; C_2 \; \cap \; \dots \dots \; \cap \; C(a)$\\
\vspace{0.1cm}\\
we have, $\begin{bmatrix}
a & b\\
b & a-b
\end{bmatrix} \; \in \; C_1$\\
\vspace{0.1cm}\\
and since belongs to $C_2$\\
$ \Longleftrightarrow \; a \; = \; a - b$\\
$\Longleftrightarrow \; b \; = \; 0$\\
$\therefore \; C_1 \cap C_2 \; = \; \left\{\begin{bmatrix}
a & 0\\
0 & a
\end{bmatrix} \; \bigg/ \; a \; \not= \; 0 \; \right\} \; = \; a I_n$\\
\vspace{0.2cm}\\
Which is $z(G)$\\
\end{enumerate}
\end{testexample}

\mybox{Note}{red!40}{red!10}{Scalar matrices are abelian \\
\& each scalar multiple is \textcolor{blue}{Center}}
\end{document}